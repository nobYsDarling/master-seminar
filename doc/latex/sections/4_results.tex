\section{Fazit}

Ziel dieser Arbeit war es ein grundlegendes Verständnis für eine mögliche Verwendung neuronaler Netze in Verbindung mit Verkehrssystemen zu erlangen. Es konnte gezeigt werden, dass die Verwendung von Hopfield-Netzen einen validen Ansatz vermuten lässt. Jedoch hat sich auch herausgestellt, dass es hier noch mehr bedarf und weitere Aspekte genauer untersucht werden müssen, um eine finale Aussage zu treffen. Hierzu gehört noch vor einer technischen Implementierung als Proof of Concept, ein Konzept zu einem Modell von Verkehrsnetzen in Verbindung mit Hopfield-Netzen und weiteres Verständnis zu dem Zusammenhang der Stabilität der in den Netzen verwendeten Energiefunktion.

Die Frage nach einer technischen Abbildung von Verkehrsnetzen in Verbindung mit Hopfield-Netzen sollte dabei folgende Fragen beantworten:

1) Wie exakt können Verkehrsetze abgebildet werden? Beachte: Fahrspuranzahl, Abbiegespuren, Rückstaus, Bedarfsampeln, Einbahnstraßen, Baustellen etc.
2) Wie granular sollten diese Modelle sein, um zum einen noch performant zu sein und zum anderen überhaupt mit daten belieferbar?

Danach müsste ein Proof of Concept erstellt werden, um das Modell zu verifizieren. Hier kann es hilfreich sein weiteres Verständnis über die Stabilitätseigenschaften zu haben, um ggf. z.B. Parameter des Netzes zu optimieren.


in die mail
Im Rahmen dieses Seminars bestand Kontakt zum Landesamt für Straßen, Brücken und Gewässer. Interesse an weiterem scheint gegeben
Daten liegen vor. 