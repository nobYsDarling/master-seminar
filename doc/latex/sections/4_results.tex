\section{Fazit}

Ziel dieser Arbeit war es ein grundlegendes Verständnis für eine mögliche Verwendung neuronaler Netze in Verbindung mit Verkehrssystemen zu erlangen. Es konnte gezeigt werden, dass die Verwendung von Hopfield-Netzen einen validen Ansatz vermuten lässt. Jedoch hat sich auch herausgestellt, dass es hier noch weiterer Arbeit bedarf und weitere Aspekte genauer untersucht werden müssen, um eine finale Aussage über die praktische Nutzbarkeit von Hopefield-Netzen in diesem Zusammenhang treffen zu können. Hierzu gehört noch vor einer technischen Implementierung als Proof of Concept, ein Konzept zu einem Modell von Verkehrsnetzen in Verbindung mit Hopfield-Netzen und weiteres Verständnis zu dem Zusammenhang der Stabilität der in den Netzen verwendeten Energiefunktion.

Die Frage nach einer technischen Abbildung von Verkehrsnetzen in Verbindung mit Hopfield-Netzen sollte dabei die Fragen beantworten "Wie exakt \textit{können} Verkehrsetze abgebildet werden?" (Dabei sind Dimensionen zu betrachten wie die Fahrspuranzahl, Darstellung von Abbiegespuren, Rückstaus, Bedarfsampeln, Einbahnstraßen, Baustellen etc.) und "Wie exakt \textit{sollten} Verkehrsetze abgebildet werden?", um performant und abbildbar zu sein.

Danach müsste ein Proof of Concept erstellt werden, um das Modell zu verifizieren. Hier kann es hilfreich sein weiteres Verständnis über die Stabilitätseigenschaften zu haben, um ggf. z.B. Parameter des Netzes zu optimieren.
